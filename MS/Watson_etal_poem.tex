% !TEX TS-program = pdflatex
\documentclass[12pt,letter]{article}

\renewcommand{\familydefault}{\sfdefault}
\setlength{\parindent}{0pt} 
\setlength{\parskip}{4ex}
\headsep = 35pt

\usepackage[margin=2.9cm]{geometry}
\usepackage[usenames,dvipsnames]{color}
\usepackage[mathlines]{lineno}
\usepackage{graphicx}
\usepackage{amssymb}
\usepackage{gensymb}
\usepackage{bibentry}
\usepackage{setspace}
\usepackage{fancyhdr}
\usepackage{lastpage} 
\usepackage{wrapfig}
\usepackage[footnotesize]{caption}
\usepackage{subfig}
\usepackage{booktabs}
\usepackage{soul}
\usepackage{epsfig}

\usepackage{cite}
\usepackage[]{natbib}

\usepackage[compact]{titlesec}
\titlespacing{\section}{0pt}{*0}{*-3}
\titlespacing{\subsection}{0pt}{*0}{*-3}
\titlespacing{\subsubsection}{0pt}{*0}{*-3}


%----------------------------------------------------------------------------------
\begin{document}
\setstretch{1.8}

\begin{center}
	{\Huge{Global Marine Tipping Points Under Climate Change}}\\
	{\color{RoyalBlue}{{In no particular order... James R. Watson$^{1,2}$, Susa Niiranen$^1$, Charles A. Stock$^3$}}}\\
\end{center}



\footnotesize{

$^1$Stockholm Resilience Centre, Stockholm University. \\
$^2$Department of Ecology and Evolutionary Biology, Princeton University, NJ 08544, USA. \\
$^3$NOAA Geophysical Fluid Dynamics Laboratory, Princeton NJ 08544, USA.\\ 
%$^*$Author for correspondence: james.watson@su.se\\
\\
\textbf{Keywords}: regime shif, tipping point, food web, fisheries, size-spectra, macroecology, climate change.

}
%----------------------------------------------------------------------------------



%----------------------------------------------------------------------------------
\setstretch{2}
\clearpage
\linenumbers

%----------------------------------------------------------------------------------
% Header
\pagestyle{fancy}
\fancyhf{}
\fancyfoot[R]{\footnotesize{Page \thepage\ of \pageref{LastPage}}}
\renewcommand{\headrulewidth}{0cm}
\lhead{\footnotesize{\textit{Marine tipping points}}}
\rhead{\footnotesize{\textit{Watson et al.}}}


%----------------------------------------------------------------------------------
\section*{ABSTRACT}

\clearpage
\section*{INTRODUCTION}

\section*{METHODS}

\subsection*{The Princeton Ocean Ecosystem Model: POEM}
The Princeton Ocean Ecosystem Model (POEM) has two main components: (1) the Carbon, Ocean Biogeochemistry and Lower Trophics or COBALT marine ecosystem model and (2) a size-based model  of high-trophic level dynamics. 

COBALT is used to provide the biogeochemical and planktonic food web dynamics for the size-based food web model herein.  COBALT uses 33 state variables to resolve global-scale cycles of nitrogen, carbon, phosphate, silicate, iron, calcium carbonate, oxygen, and lithogenic material \citep{Stock:2013ek}. COBALT is run as part of the Modular Ocean Model (MOM) version 4.1 \citep{SM:2012wp}, with 60 year simulations (1948-2008) forced by the Common Ocean-Ice Reference Experiment (CORE-II) data set \citep{Large:2008kb} as well as a future scenario running from 2006 to 2100 using relative concentration pathways (rcp) 8.5 as the carbon emissions scenario.

The horizontal resolution of the simulation is $1^o$ Latitude/Longitude, except along the equator where the resolution is refined to $1/3^o$. The model uses 50 vertical layers, with a resolution of 10m over the top 200m. The representation of planktonic food web dynamics within COBALT includes small and large phytoplankton � with the latter group comprised of diatoms and dinoflagellates, and three zooplankton groups that feed on phytoplankton, bacteria and each other according to mean predator prey size ratios \citep[e.g.][]{Hansen:1994td}. The ``small'' zooplankton group represents microzooplankton that are $< 200 \mu m$ in equivalent spherical diameter (ESD).  The medium zooplankton are parameterized as small to medium bodied copepods ($0.2-2 mm$ ESD), and the large zooplankton are parameterized as large copepods/krill ($2-20 mm$ ESD). The parameterization of trophic interactions relies primarily on the allometric and bioenergetics relationships described in \citep[refs.][]{Hansen:1994td, Hansen:1997vn, Straile:1997ta}, and the model was calibrated to ensure quantitative consistency with large-scale planktonc food web dynamics, including patterns in primary and meso-zooplankton production \citep{Stock:2013ek}. COBALT estimates zooplankton mortality rates using a density dependent closure term \citep{Stock:2013ek}. This mortality term accounts for all zooplankton production that is not consumed by other zooplankton, and hence represents feeding by higher predators and other sources of mortality that are not resolved explicitly. This density dependence has empirical support \citep[e.g.][]{Ohman:2001fl} and reflects an assumption that the biomass of predators responds to the biomass of prey. 

COBALT is linked to the size-based food web model described herein in an ``off-line'' fashion. This is done at the phytoplankton level: information about phytoplankton from COBALT is used to drive the dynamics of the size-based food web model. Like COBALT, the size-based food web model estimates the biomasses of meso-zooplankton, and their feeding rates on phytoplankton. Hence, we have two independent measures of zooplankton dynamics, one from COBALT and one from the size-based food-web model. As a consequence it is possible to force the size-based model's  zooplankton feeding rates to match COBALT's. 

The size-based food web model is comprised to four major animal groups: zooplankton, a planktivorous fish, a piscivorous fish and benthic invertebrates. Zooplankton are modeled a simple biomass pools, within an inflow of biomass dictated by COBALT's fluxes. Hence, in the size-based model we do not model zooplankton feeding on phytoplankton mechanistically (i.e. with a Type II feeding function as is done in COBALT). However, by tracking the biomass of zooplankton in both the size-based food-web model and COBALT, measures of off-line coupling accuracy can be made. Simply put, the quotient of zooplankton biomass from COBALT and the size-based food-web model.

Population dynamics for the remaining groups - piscivore, planktovore and benthic invertebrate - are modeled in a more ecologically explicit manner, following the \textit{size-structured} formulation of Van Leeuwan et al. 2008. The population of each group is modeled as a collection of size-classes, with smaller classes growing (somatically) into larger size classes. The size-structured model is general, being applied to each group with no difference in model-structure. It is mainly the choice of parameters that defines the different groups.

The rate of change of biomass density ($g$ $m^{-2}$) for the smallest size classes in either the piscivore, planktivore or benthic groups is: 
\begin{equation}
\frac{d B_i}{dt} =  \sum_{j\neq i}^{N-1}(1-\kappa_j)\nu_j B_j - \kappa_i \gamma_i B_i - d_i B_i,
\end{equation}
where $i = 1$, the index of the smallest size-class in a given group. The first term on the right-hand-side is the total amount of newborn fish or crustacean biomass (i.e. summed over all larger size classes) recruiting to the smallest size-class. $\nu_i$ is the total energy available for growth and $\kappa$ is a unit-less parameter that controls the fraction of $\nu$ used for somatic growth, hence $1-\kappa$ is the energy invested in the production of larvae/eggs. The total amount of newborn biomass recruiting to the first size-class is the summation of larvae/egg production over all larger size-classes (index $j$ not $i$ to $N$ the total number of size-classes in a given group). The second term is the loss of biomass due to somatic growth or maturation, where $\gamma$ is a function of $\nu$ that accounts for the loss of biomass within a size-class. The last term is the loss of biomass due to natural mortality. 

For any larger size class of any group, the rate of change of biomass is:  
\begin{equation}
\frac{d B_i}{dt} = \kappa_{i-1} \gamma_{i-1} B_{i-1} - \kappa_i \gamma_i B_i - (1-\kappa_i) \nu_i B_i - d_i B_i,
\end{equation}
for $i=2,...,N$. Here, the first term on the right-hand-side is maturing biomass from the previous size-class, the second term is the loss of biomass due to somatic growth, the third term is the loss of biomass to egg/larvae production and the last term is the loss of biomass due to natural mortality. Note, there is no recruitment term, for we assume all newborn biomass recruits to the smallest size-class.

The energy available for growth is defined for a given size-class as: 
\begin{equation}
\nu_i = \lambda I_i - T_i,
\end{equation}
where $\lambda$ is the food assimilation efficiency ($g_i$ $g_j^{-1}$), $I$ is the biomass consumed ($g_j$ $g_i$ $d^{-1}$) and $T$ is biomass-specific metabolic costs ($g_i$ $g_i^{-1}$ $d^{-1}$). The energy available for somatic growth is:
\begin{equation}
\gamma_i = \frac{\nu_i - \frac{1}{\kappa_i}d_i}{1-z_i^{(1-d_i) / (\kappa_i \nu_i)}},
\end{equation}
where $d_i$ is the natural mortality rate of size-class $i$ ($g_i$ $g_i^{-1}$ $d^{-1}$) and $z_i$ is the ratio of the initial and the final body size that a particular life stage encompasses and hence reflects the size range that an individual has to grow through before maturing to the next stage.


\subsubsection*{Consumption}
Consumption of prey biomass is dealt with in slightly different ways for the piscivore, planktivore and benthic groups. The piscivore is able to feed on all groups, including itself, and the biomass specific feeding rate of any given piscivore size-class $i$ is modeled using a multi-prey Type II feeding function:
\begin{equation}
I_i = \sum_j^N \frac{a_i B_j \phi_{ij}}{1+a_i \tau_i \sum_j^N B_j \phi_{ij}},
\end{equation}
where $a_i$ is the size-class specific per unit body-mass search rate ($m^2$ $d^{-1}$ $g_i^{-1}$), $\tau_i$ is the time it takes for one individual of size-class $i$, in terms of body-weight, to digest a unit of biomass of prey $j$ ($d$ $g_j$ $g_i$). Here, $j$ indexes all size-classes in all groups, and it is a diet-preference factor $\phi_{ij}$ that determines how much of any given prey, piscivore $i$ eats. Diet preferences are defined using empirical predator-prey mass ratios (PPMRs). Specifically, $PPMR$ values are taken from the empirical gut-content analysis of \citet[][]{Barnes:2010vg}, who defined a mean PPMR $\psi = 3$, with a PPMR standard deviation $\sigma = 1.3$. Similar to \citet{Blanchard:2009il, Blanchard:2012bp}, $\phi_{ij}$ values were then assumed to be a Gaussian function of the logarithm of predator and prey body masses, $\hat{s_i}$ and $\hat{s_j}$ respectively, with a maximum value when the PPMR is $\psi$ and a standard deviation $\sigma$: $\phi_{ij} = 1/(\sigma \sqrt{2 \pi})\cdot exp(-(\hat{s_i} - \hat{s_j} - \psi) / 2\sigma^2)$ when $\hat{s_i} - \hat{s_j} > 0$, otherwise $\phi_{ij} = 0$. 

%%% Needs to have diet preference
In contrast to the piscivore, the planktivore eats only zooplankton: 
\begin{equation}
I_i = \frac{a_i (Z_1 + Z_2)}{1+a_i \tau_i (Z_1 + Z_2)},
\end{equation}
where in this case $i$ indexes only the planktivore size-classes, and it is evident that we assume all planktivore size-classes eat all zooplankton, both medium and large, without preference. 

%%% Detritivores eat themselves, need to account for that.
The benthic invertebrate feeds on detrital matter accumulating on the sea-floor, which is modeled explicitly:
\begin{equation}
\frac{dW}{dt} = D - S - \sum_i^N I_i,
\end{equation}
where $W$ is the biomass pool of detrital matter on the sea-floor ($g$ $m^{-2}$), $D$ is the flux of detrital matter from the water column ($g$ $m^{-2}$ $d^{-1}$; this is given by COBALT as an offline component), $S$ is the rate of sedimentation of this biomass ($g$ $m^{-2}$ $d^{-1}$; again a rate given by COBALT) and $I_i$ is the consumption by the detritivore of size-class $i$ of which there are $N$. Like consumption elsewhere, a Type II feeding function is used:
\begin{equation}
I_i = \frac{a_i W}{1 + a_i \tau_i W},
\end{equation}
where $i$ indexes only the detritivore size-classes. 


\subsection*{Offline coupling and Numerical Integration}

\subsection*{Parameterization}
The list of 


\section*{RESULTS}

\section*{DISCUSSION}


\section*{ACKNOWLEDGEMENTS}
	
\bibliographystyle{apalike}
\bibliography{mybib}


%----------------------------------------------------------------------------------
%\clearpage
%\begin{figure}[htbp]
%\begin{center}
%\noindent\includegraphics[width=1\textwidth]{../../Figs/PDF/Fig_1}
%\caption{(a \& b) Average medium and large zooplankton biomass densities (zm and zl; $g$ $m^{-3}$) and (c \& d) average medium and large zooplankton mortality rates (zm mort and zl mort; $g$ $m^{-3}$ $day^{-1}$), for the period 1997-2007. Note, white areas do not denote zero values, rather those that are very low (see color bars)}.
%\label{Fig 1}
%\end{center}
%\end{figure}

%----------------------------------------------------------------------------------
\clearpage
\section*{Tables}
% Table of parameters
\vspace{2em}
\begin{table}[htbp]\footnotesize
   \centering
   %\topcaption{Table captions are better up top} % requires the topcapt package
   \begin{tabular}{@{} lllr @{}} % Column formatting, @{} suppresses leading/trailing space
      \toprule
      %\multicolumn{2}{c}{Item} \\
      Parameters:     & Description & Example values & Units \\
      \cmidrule(r){1-2} 
      $s_{i}$      & body mass of predator $i$ & 1,100,1000 & $g$ \\
      $s_{j}$      & body mass of preferred prey $j$ & 0.001,0.1, 1 & $g$ \\
      $l_{i}$      & body length of $i$ & 0.03,0.12,0.25 & $m$ \\
      $\nu_i$ & $i$'s swimming speed & 1700,5000,8500 & $m$ $day^{-1}$\\      
      $V_i$  & volumetric search rate of $i$ &3.8, 1.8, 1.2 & $m^{3} day^{-1} g^{-1}$ \\
      $\phi_{ij}$   & vulnerability of prey & [0,1] & $-$ \\
      $\tau_{ij}$ & time for $i$ to handle $j$ & 5.7x$10^{-5}$, 0.028, 0.56 &  $days$ \\

      \\      
      Variables:    & & & \\
      \cmidrule(r){1-2} 
      $B_i$      & biomass density of size-class $i$ & -- & $g$ $m^{-3}$ \\
      $c_i$      & consumption rate of $i$ & -- & $day^{-1}$ \\     
      $d_i$      & predation mortality rate of $i$ & -- & $day^{-1}$ \\     
      $m_i$     & metabolic rate of $i$ & 0.006, 0.002, 0.001 &  $day^{-1}$ \\          
      $\mu_i$  & natural mortality rate of $i$ &0.006, 0.0002, 0.0001 & $day^{-1}$ \\     
      $J_i$  & flux of size-class $i$ & -- & $g$ $m^{-3}$ $day^{-1}$ \\     
     
       \\
      Constants:    &  &  &\\
      \cmidrule(r){1-2} 
      $\lambda$ & Conversion efficiency & 0.7 & \% \\
      $N$ & Number of size classes & 15 & \# \\    
      
      \bottomrule   
   \end{tabular}
   \caption{Table of model variables, parameters and constants posed in terms of size classes $i$ and $j$, a predator and its most preferred prey respectively. Values were chosen for a 1g, 100g and 1000g sized predator, and all values are calculated assuming 12$^oC$ temperature. Handling times were calculated using the formula in \citet{Rall:2012ir}.}
   \label{tab:booktabs}
\end{table}



%----------------------------------------------------------------------------------
%\makeatletter 
%\renewcommand{\thefigure}{A\@arabic\c@figure} 
%\renewcommand{\thetable}{A\@arabic\c@table} 
%\makeatother
%\setcounter{figure}{0}

%\clearpage
%\section*{Appendixes}
%\begin{figure}[htbp]
%%\hspace{12ex}
%\noindent\includegraphics[scale = .8]{../../Figs/PDF/Fig_A1_move}
%\caption{Average biomass densities over the period 1997-2007 for forage fish (FF) and top-predators (TP). Data are produced from simulations with different movement rules: with advection only (a \& b), with net growth following movement (c \& d) and prey biomass following movement (e \& f).}
%\label{Fig S4}
%\end{figure}
%
\end{document}














